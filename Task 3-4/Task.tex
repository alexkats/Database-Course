\documentclass[12pt,a4paper,oneside]{article}

\usepackage[T2A]{fontenc}
\usepackage[english,russian]{babel}
\usepackage[utf8]{inputenc}

\begin{document}

\hfill \textbf{Кацман Алексей}

\hfill \textbf{M3438}

\bigskip

\section{ДЗ 3}
\subsection{Атрибуты}

\medskip

StudentId, StudentName, GroupId, GroupName, CourseId, CourseName, LecturerId, LecturerName, Mark.

\subsection{Функциональные зависимости}

\medskip

StudentId $\rightarrow$ StudentName

\noindent GroupId $\rightarrow$ GroupName

\noindent CourseId $\rightarrow$ CourseName

\noindent LecturerId $\rightarrow$ LecturerName

\noindent StudentId, GroupId, CourseId $\rightarrow$ Mark

\noindent StudentId, GroupId, CourseId, LecturerId $\rightarrow$ Mark

\medskip

Будем считать, что:

\begin{itemize}

\item По студенту нельзя определить группу, так как она могла смениться.

\item По курсу нельзя определить преподавателя, так как по одному курсу их может быть несколько.

\end{itemize}

\medskip

Также заметим, что последнее правило можно выкинуть, так как если есть функциональная зависимость $X\ \rightarrow\  Y$, то есть и функциональная зависимость $X\ +\ \{a\}\ \rightarrow\ Y$.

\subsection{Ключи}

\medskip

$\{$StudentId, GroupId, CourseId, LecturerId$\}$ является ключом, так как $\{$StudentId, GroupId, CourseId, LecturerId$\}^{+}$ содержит множество всех атрибутов.

\subsection{Неприводимое множество функциональных зависимостей}

\medskip

\begin{itemize}

\item Сначала расщепим правые части:

\medskip

StudentId $\rightarrow$ StudentName

GroupId $\rightarrow$ GroupName

CourseId $\rightarrow$ CourseName

LecturerId $\rightarrow$ LecturerName

StudentId, GroupId, CourseId $\rightarrow$ Mark

\medskip

\item Теперь будем минимизировать по включению левые части.

\medskip

$\{$StudentId, GroupId, CourseId$\}^{+}$

\begin{itemize}

\item $\{$StudentId, GroupId, CourseId$\}$

\item $\{$StudentId, GroupId, CourseId, StudentName, GroupName, CourseName, Mark$\}$

\end{itemize}

\medskip

\begin{enumerate}

\item $\{$StudentId, GroupId$\}^{+}$

\begin{itemize}

\item $\{$StudentId, GroupId$\}$

\item $\{$StudentId, GroupId, StudentName, GroupName$\}$

\end{itemize}

\item $\{$StudentId, CourseId$\}^{+}$

\begin{itemize}

\item $\{$StudentId, CourseId$\}$

\item $\{$StudentId, CourseId, StudentName, CourseName$\}$

\end{itemize}

\item $\{$GroupId, CourseId$\}^{+}$

\begin{itemize}

\item $\{$GroupId, CourseId$\}$

\item $\{$GroupId, CourseId, GroupName, CourseName$\}$

\end{itemize}

\end{enumerate}

\medskip

$\{$StudentId, GroupId, CourseId$\}$ - минимальное по включению.

\medskip

\item Теперь будем минимизировать множество функциональных зависимостей

\begin{enumerate}

\item $\{$StudentId $\rightarrow$ StudentName$\}$ убрать не можем, так как иначе никак не вывести StudentName

\item $\{$GroupId $\rightarrow$ GroupName$\}$ убрать не можем, так как иначе никак не вывести GroupName

\item $\{$CourseId $\rightarrow$ CourseName$\}$ убрать не можем, так как иначе никак не вывести CourseName

\item $\{$StudentId, GroupId, CourseId $\rightarrow$ Mark$\}$ убрать не можем, так как иначе никак не вывести Mark

\end{enumerate}

\medskip

В итоге получим следующее неприводимое множество функциональных зависимостей:

\begin{itemize}

\item StudentId $\rightarrow$ StudentName

\item GroupId $\rightarrow$ GroupName

\item CourseId $\rightarrow$ CourseName

\item StudentId, GroupId, CourseId $\rightarrow$ Mark

\end{itemize}

\end{itemize}

\end{document}
